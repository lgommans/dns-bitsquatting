\documentclass[conference]{IEEEtran}
\usepackage{float}
\usepackage{graphicx}
\usepackage[pdftex,
			%final,
            pdfauthor={Adrien Raulot and Luc Gommans},
            pdftitle={TODO},
            pdfkeywords={TODO}]{hyperref}
\usepackage{wrapfig}
\usepackage{booktabs}
\hypersetup{
	colorlinks=true,       % false: boxed links; true: coloured links
	linkcolor=blue,        % colour of internal links
	citecolor=blue,        % colour of links to bibliography
	filecolor=magenta,     % colour of file links
	urlcolor=blue
}

\usepackage[
    backend=bibtex,
    style=numeric,
    sorting=none,
    maxcitenames=2
]{biblatex}

\bibliography{references.bib}

\begin{document}

\title{DNS bitsquatting\\\vspace{5mm} \large  \today}
\author{
\IEEEauthorblockN{Adrien Raulot}
\IEEEauthorblockA{
Security and Network Engineering \\
University of Amsterdam \\
adrien.raulot@os3.nl}
\and
\IEEEauthorblockN{Luc Gommans}
\IEEEauthorblockA{
Security and Network Engineering \\
University of Amsterdam \\
os3-ot@lucgommans.nl}
}
\maketitle
\thispagestyle{plain}
\pagestyle{plain}

\begin{abstract}

	In this paper we will investigate the current state of random DNS bit flips.
	While random bit errors in a machine are not very common, the number of DNS
	queries performed worldwide is extremely large. This might lead one to query
	for a domain name, and subsequently connect to the resulting IP address, which
	one never intended to connect to.

\end{abstract}

\section{Introduction}

The vast majority of internet-connected computers perform DNS requests during
conventional use. Because the requests are so ubiquitous, there is reason to
suspect that sometimes, due to random variations, errors might naturally occur.
When an error occurs, one might query for and subsequently connect to a domain
which one never intended to communicate with.

In this paper we will investigate likely causes for bit flips in desktop
computers and attempt to determine whether the resulting attack, bitsquatting,
is still viable.

%TODO: insert a disorderly substitute for a table of contents here, because that's how Science(TM) likes it.


\section{Research Questions}\label{sec:researchq}

Our main research question is:
{\it Is bitsquatting a viable attack?}

\vspace{0.1cm}

\noindent{} This main research question divides in four sub-questions:

\begin{enumerate}
    \item What could a bit-squatting attack be used for?
	\item What circumstances cause random bit errors?
	%\item Which kinds of devices are susceptible to random bit errors? TODO: uncomment if applicable
	\item Are random bit errors currently occurring in the IT ecosystem? %TODO: better phrasing for 'in the wild'
	\item Are organisations mitigating the risk by pre-emptively purchasing
	      bitsquat domains?
\end{enumerate}


\section{Ethical Considerations}\label{sec:ethics}

Most of our work will focus on attempting to find causes of random bit errors.
However, in order to learn whether random bit errors are currently occuring in
practice, we will need to work with real domains and user data. This data is
kept confidential and will be destroyed after completion of the project. The
domains we register will not be used for anything other than determining
whether this is the result of a random bit error.


\section{Related Work}\label{sec:relwork}

In 2011, Artem Dinaburg presented for the first time the bitsquatting attack at
the DEFCON 19 security conference. In his
whitepaper\cite{dinaburg2011bitsquatting}, he describes bitsquatting as an
attack relying on random bit errors to redirect connections intended for
popular domains to domains registered by malicious entities. Since 2011, only
very few research papers have been published about bitsquatting, therefore the
purpose of this research. Another research from 2011 by Nick et
al.\cite{nikiforakis2013bitsquatting} showed that bitsquatting was still
popular by monitoring newly registered bitsquatting domains over a period of
270 days. In this period, a total of 5,366 different bitsquatting domains
targeting 491 out of the Alexa top 500 domains were registered.

\section{Methods}\label{sec:method}

Although random bit flips are theoretically possible, the unlikeliness of a bit flip in
practice is important. Bit flips or soft errors are more likely to occur in certain
circumstances such as high temperature, low voltage, high speed, alpha
particles, cosmic rays, and others. This not only may have an impact on the
applications running on a system but also raises security concerns as attacks
such as the bitsquatting attack we describe in this paper may take advantage of
soft errors. During this research, we mainly focused on producing soft errors
in memory modules, although we also tested and monitored the hard disk and the network for
bit flips. Nowadays, Error-Correcting Code memory is commonly used in most
computers where data corruption cannot be tolerated.

Firstly, we acquired an experimental desktop [TODO: talk about the server also?] and implemented three types of test program to monitor bit flips:

\begin{itemize}
  % TODO: For each, quote appendix
  \item Memory: [TODO: describe program]
  \item Disk: [TODO: describe program]
  \item Network: [TODO: describe program]
\end{itemize}

Secondly, we registered and monitored 4 bitsquatting domains:

\begin{itemize}
  \item eoogle.be
  \item eoogle.nl
  \item whatsapx.net
  \item whatsapx.co.uk
\end{itemize}

\begin{table}[H]
  \centering
  \caption{Bitsquatting vs original domain names}
  \label{table-bits}
  \begin{tabular}{|l|l|l|}
    \hline
    \textbf{Domain}   & \textbf{Bits} \\ \hline
    eoogle   & 01100101... \\ \hline
    google   & 011001\textbf{1}1... \\ \hline
    whatsapx & ...01111000 \\ \hline
    whatsapp & ...0111\textbf{0}000 \\ \hline
   \end{tabular}
\end{table}

\subsection{Experimental Setup}\label{sec:setup}

From the different fault injection techniques known to
exist\cite{barenghi2012fault}, we decided to use heat. This technique is
realistic of situations where desktop computers, servers and smartphones are
impacted by high temperatures and it is a relatively cheap and easy technique
to set up. Although this technique may possibly damage the hardware, we were
careful to place the disposable hardware in a safe environment and constantly
measure the temperature. The latter however has been more difficult than we
expected, as we were not able to retrieve thermal values from the system. To
circumvent this, we monitored the hard drive temperature as an indactor and
manually measured the temperature of the CPU and memory with a infrared
temperature sensor gun.

In order to heat up the memory, disk and CPU while letting the tests (described
in section \ref{sec:method}) run, the first step was to cover the air inputs
while always keeping the computer case closed and taking care of letting air
flow through the Power Supply Unit (PSU). Since this had a very low impact on
the components temperature, next step was to cover the computer with a cloth to
prevent heat from dissipating. Although this had a bigger impact on the overall
computer internals temperature, we were not able to produce any bit flips.
Therefore, we eventually ended up manually heating the memory modules with a
hair dryer. Hereafter is a table representing stepstones in the temperature
increase during the experiments:

%TODO: Very basic for now, more for clear visualization, feel free to modify / add
\begin{table}[H]
  \centering
  \caption{Temperature levels}
  \label{table-temp}
  \begin{tabular}{|l|l|l|}
    \hline
    \textbf{Component}   & \textbf{Min. (Celsius)} & \textbf{Max. (Celsius)}\\ \hline
    CPU   & ? & 100? \\ \hline
    Memory & ? & 90? \\ \hline
    Disk & 40? & 60? \\ \hline
   \end{tabular}
\end{table}


Server:
- DDR2 ECC memory

Desktop:
- 2 * 2 GB of DDR non-ECC memory


%TODO: *Rowhammer

\section{Discussion}\label{sec:disc}

TODO


\section{Conclusion}\label{sec:conc}

TODO


\section{Future Work}\label{sec:futwork}

TODO


\printbibliography

\end{document}

