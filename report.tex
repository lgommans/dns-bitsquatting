\documentclass[conference]{IEEEtran}
\usepackage{float}
\usepackage{graphicx}
\usepackage[pdftex,
			%final,
            pdfauthor={Adrien Raulot and Luc Gommans},
            pdftitle={TODO},
            pdfkeywords={TODO}]{hyperref}
\usepackage{wrapfig}
\usepackage{booktabs}
\hypersetup{
	colorlinks=true,       % false: boxed links; true: coloured links
	linkcolor=blue,        % colour of internal links
	citecolor=blue,        % colour of links to bibliography
	filecolor=magenta,     % colour of file links
	urlcolor=blue
}

\usepackage[
    backend=bibtex,
    style=numeric,
    sorting=none,
    maxcitenames=2
]{biblatex}

\bibliography{references.bib}


%TODO: 'bit-squatting' or 'bit squatting'?

\begin{document}

\title{title goes here \\\vspace{5mm} \large  \today}
\author{
\IEEEauthorblockN{Adrien Raulot}
\IEEEauthorblockA{
Security and Network Engineering \\
University of Amsterdam \\
adrien.raulot@os3.nl}
\and
\IEEEauthorblockN{Luc Gommans}
\IEEEauthorblockA{
Security and Network Engineering \\
University of Amsterdam \\
os3-ot@lucgommans.nl}
}
\maketitle
\thispagestyle{plain}
\pagestyle{plain}

\begin{abstract}

	In this paper we will investigate the current state of random DNS bit flips.
	While random bit errors in a machine are not very common, the number of DNS
	queries performed worldwide is extremely large. This might lead one to query
	for a domain name, and subsequently connect to the resulting IP address, which
	one never intended to connect to.

\end{abstract}

\section{Introduction}

The vast majority of internet-connected computers perform DNS requests during
conventional use. Because the requests are so ubiquitous, there is reason to
suspect that sometimes, due to random variations, errors might naturally occur.
When an error occurs, one might query for and subsequently connect to a domain
which one never intended to communicate with.

In this paper we will investigate likely causes for bit flips [TODO: in
different devices] and attempt to determine whether the resulting attack,
bit-squatting, is still viable.

%TODO: insert a disorderly substitute for a table of contents here, because that's how Science(TM) likes it.


\section{Research Questions}\label{sec:researchq}

Our main research question is:
{\it Is bit-squatting a viable attack?}

\vspace{0.1cm}

\noindent{} This main research question divides in four sub-questions:

\begin{enumerate}
    \item What could a bit-squatting attack be used for?
	\item What circumstances cause random bit errors?
	%\item Which kinds of devices are susceptible to random bit errors? TODO: uncomment if applicable
	\item Are random bit errors currently occurring in the wild? %TODO: better phrasing for 'in the wild'
	\item Are organisations mitigating the risk by pre-emptively purchasing
	      bit-squat domains?
\end{enumerate}


\section{Ethical Considerations}\label{sec:ethics}

Most of our work will focus on attempting to find causes of random bit errors.
However, in order to learn whether random bit errors are currently occuring in
practice, we will need to work with real domains and user data. This data is
kept confidential and will be destroyed after completion of the project. The
domains we register will not be used for anything other than determining
whether this is the result of a random bit error.


\section{Related Work}\label{sec:relwork}

TODO


\section{Methods}\label{sec:method}

TODO


\subsection{Experimental Setup}\label{sec:setup}

TODO


\section{Discussion}\label{sec:disc}

TODO


\section{Conclusion}\label{sec:conc}

TODO


\section{Future Work}\label{sec:futwork}

TODO


\printbibliography

\end{document}

