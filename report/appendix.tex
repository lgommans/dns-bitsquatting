\begin{appendices}

\section{Memory test}\label{Apdx-memory}
\begin{lstlisting}[language=C]
#include <stdio.h>
#include <stdlib.h>
#include <string.h>
#include <sys/types.h>
#include <sys/stat.h>
#include <fcntl.h>
#include <unistd.h>

#define RANDOM_SRC "/dev/urandom"
#define P_LENGTH 4 * 1000
#define SET_ITERATIONS 1
#define READ_ITERATIONS 1

void prnt(char *s, int l) {
  for (int i = 0; i < l; i++)
    printf("%x ", s[i]);
}

int main(int argc, char** argv) {
  long alloc_size = strtol(argv[1], NULL, 10);
  char *s = (char*)malloc(sizeof(char) * alloc_size);
  char p1[P_LENGTH], p2[P_LENGTH], p3[P_LENGTH], pa[P_LENGTH];
  int i = 0;
  long shift = 0;
  int rdm = open(RANDOM_SRC, O_RDONLY);
  if (rdm < 0) {
    return 1;
  }

  if (read(rdm, pa, P_LENGTH) < 0) {
    return 3;
  }

  while (1) {
    if (shift + P_LENGTH >= alloc_size) {
      shift = 0;
      if (i == SET_ITERATIONS + READ_ITERATIONS) {
        printf("New random values\n");
        i = -1;
        if (read(rdm, p1, P_LENGTH) < 0 ||
            read(rdm, pa, P_LENGTH) < 0) {
          return 2;
        }
        memcpy(p2, p1, P_LENGTH);
        memcpy(p3, p1, P_LENGTH);
      }
      else {
        if (i == SET_ITERATIONS) {
          printf("Set round complete. Reading...\n");
        }
      }
      i++;
    }

    if (i < SET_ITERATIONS) {
      memcpy(s + shift, pa, P_LENGTH);
      memcpy(s + shift, p1, P_LENGTH);
    }
    else {
      if (memcmp(s, p1, P_LENGTH)) {
        printf("Bit flip detected. Should have been:\n");
        prnt(p1, P_LENGTH);
        printf("\nBut was instead:\n");
        prnt(s, P_LENGTH);
        printf("\n");
        if (memcmp(p1, p2, P_LENGTH) || memcmp(p2, p3, P_LENGTH) || memcmp(p1, p3, P_LENGTH)) {
          printf("And the pattern was corrupted.\n");
          continue;
        }
      }
    }
    shift += P_LENGTH;
  }
  free(s);
  close(rdm);
}
\end{lstlisting}

\section{Disk test}\label{Apdx-disk}
\begin{lstlisting}[language=Python]
#!/usr/bin/env python3

path = '/dev/sdd'
filesize = 400e9
patternlength = 4096 - 1


import os, binascii

testrunnum = 0

while True:
  if testrunnum % 4 == 0:
    print('Generating new pattern.')
    pattern = os.urandom(patternlength)

  testrunnum += 1

  f = open(path, 'wb')
  wrotebytes = 0
  while wrotebytes < filesize - patternlength:
    f.write(pattern)
    wrotebytes += patternlength
  f.close()

  f = open(path, 'rb')
  readbytes = 0
  while readbytes < filesize - patternlength:
    r = f.read(patternlength)
    readbytes += patternlength
    if r != pattern:
      print('Changed! Original:')
      print(binascii.hexliy(pattern))
      print('Result:')
      print(binascii.hexliy(r))
      print('-------------')

  print('Test run {} complete.'.format(testrunnum))
\end{lstlisting}

\section{Network test}\label{Apdx-network}
\subsection{Transmitter}
\begin{lstlisting}[language=C++]
#include <sys/socket.h>
#include <netdb.h>
#include <memory.h>
#include <iostream>
#include <ctime>
#include <openssl/sha.h>
#include <sys/stat.h>
#include <fcntl.h>
#include <unistd.h>
#include <stdlib.h>

using namespace std;

void sha256(const unsigned char* str, int len, unsigned char* hash)
{
  SHA256_CTX sha256;
  SHA256_Init(&sha256);
  SHA256_Update(&sha256, str, len);
  SHA256_Final(hash, &sha256);
}

int resolvehelper(const char* hostname, int family, const char* service, sockaddr_storage* pAddr)
{
  int result;
  addrinfo* result_list = NULL;
  addrinfo hints = {};
  hints.ai_family = family;
  hints.ai_socktype = SOCK_DGRAM; // without this flag, getaddrinfo will return 3x the number of addresses (one for each socket type).
  result = getaddrinfo(hostname, service, &hints, &result_list);
  if (result == 0)
  {
    //ASSERT(result_list->ai_addrlen <= sizeof(sockaddr_in));
    memcpy(pAddr, result_list->ai_addr, result_list->ai_addrlen);
    freeaddrinfo(result_list);
  }

  return result;
}

int main(int argc, char** argv) {
  int evoevery = 60;  // Evolve (renew payload) every this many seconds
  int sendbytes = 1200;  // How many bytes UDP payload? (Hash is put in the last sendbytes-hashlen_bytes bytes.)
  int hashlen_bytes = 32;  // Length (in bytes) of the chosen hashing algorithm
  int errlimit = 5000;  // Die after this many errors
  int sleepDivider = 50;  // Only sleep every N transmissions, because the nanosleep call has an amazing amount of overhead, for something claiming to sleep some 'nanoseconds'...
  //int overhead = 18 + 20 + 8;  // Overhead for each transmitted packet, in bytes (eth,ip,udp)
  double bwcorrectionfactor = 0.93;

  if (argc != 4) {
    cout << "Usage: testnet <dst host> <port> <bandwidth-in-mbps>" << endl;
    return 1;
  }

  int mbitpersec = strtol(argv[3], NULL, 10);  // Bandwidth limit, in megabits. Note that tx rate is only evaluated once per evolution.
  int bwlimit = mbitpersec * 1e6 / 8;
  int errs = 0;
  int nextevo = 0;
  int result = 0;
  int sock = socket(AF_INET, SOCK_DGRAM, 0);

  sockaddr_in addrListen = {}; // zero-int, sin_port is 0, which picks a random port for bind.
  addrListen.sin_family = AF_INET;
  result = bind(sock, (sockaddr*)&addrListen, sizeof(addrListen));
  if (result == -1)
  {
    cout << "error n2";
    return 2;
  }

  sockaddr_storage addrDest = {};
  result = resolvehelper(argv[1], AF_INET, argv[2], &addrDest);
  if (result != 0)
  {
    cout << "error n3";
    return 3;
  }

  unsigned char* msg = new unsigned char[sendbytes];
  unsigned char hash[hashlen_bytes];
  int rdm = open("/dev/urandom", O_RDONLY);
  int bytessent;
  int evolution = 0;
  size_t ads = sizeof(addrDest);
  int rounds = 0;
  int i;
  double bytespersec;
  struct timespec* nsleepshit = (struct timespec*) NULL;

  struct timespec usleeptime = {0};
  usleeptime.tv_sec = 0;
  usleeptime.tv_nsec = 1e9 * (sendbytes / (double)bwlimit * sleepDivider);
  cout << "Initial nsleeptime = " << usleeptime.tv_nsec << "ns" << endl;

  while (true) {
    int t = time(NULL);
    if (t > nextevo) {
      if (rounds > 0) {
        bytespersec = (long)rounds * sendbytes / (double)evoevery * bwcorrectionfactor;
        usleeptime.tv_nsec *= bytespersec / bwlimit;
        cout << "Calculated bw in mbit/s = " << bytespersec / 1e6 * 8 << "; nsleeptime *= " << bytespersec / bwlimit << "; new nsleeptime = " << usleeptime.tv_nsec << endl;
      }

      read(rdm, msg, sendbytes - hashlen_bytes);
      msg[0] = evolution % 256;
      sha256(msg, sendbytes - hashlen_bytes, hash);
      for (i = 0; i < hashlen_bytes; ++i) {
        msg[sendbytes - hashlen_bytes + i] = hash[i];
      }

      cout << "Did " << rounds << " rounds. Starting evolution " << evolution << endl;

      rounds = 0;
      evolution++;
      nextevo = t + evoevery;
    }

    bytessent = sendto(sock, msg, sendbytes, 0, (sockaddr*)&addrDest, ads);
    if (bytessent != sendbytes) {
      cout << "WARN: bytessent = " << bytessent << endl;
      if (errs > errlimit) {
        return 4;
      }
      errs++;
    }

    rounds++;
    if (rounds % sleepDivider == 0) {
      nanosleep(&usleeptime, nsleepshit);
    }
  }

  close(rdm);
  return 0;
}
\end{lstlisting}

\subsection{Receiver}
\begin{lstlisting}[language=Python]
#!/usr/bin/env pypy

import sys, socket, hashlib, binascii

hashlen_bytes = 32  # The length of the chosen message digest, in bytes
msglen = 1200  # Bytes that should be in each packet
errlimit = 5000  # Die after this many errors

sock = socket.socket(socket.AF_INET, socket.SOCK_DGRAM)
sock.bind(('', int(sys.argv[1])))

lastevo = None
lastdata = None
rounds = 0
errs = 0

def err():
  global errs
  if errs > errlimit:
    print('Exiting due to too many errors.')
    exit(5)
  errs += 1

while True:
  data = sock.recv(msglen)
  if len(data) != msglen:
    print('WARN: data length incorrect.')
    err()

  if data[0] != lastevo:
    print('New evo after {} rounds.'.format(rounds))
    rounds = 0

  if data[0] != lastevo or not data == lastdata:
    if data[-hashlen_bytes:] == hashlib.sha256(data[:-hashlen_bytes]).digest():
      if data != lastdata and data[0] == lastevo:
      print('Note: new data, same evo, but correct hash.')

    lastdata = data
    lastevo = data[0]
  else:
    err()
    print('Hash error in data:')
    print(binascii.hexlify(data))
    if data[0] == lastevo:
      print('Correct was:')
      print(binascii.hexlify(lastdata))
    else:
      lastdata = None

rounds += 1
\end{lstlisting}

\end{appendices}
